
\section{Starter File}

\subsection{Reading the Manual's format}
SS3 begins by reading the file starter.ss. The starter file contains need information on the names of the control and data files, run conditions, and output specifications. The term COND appears in the ``Typical Value'' column of this documentation (it does not actually appear in the model files), it indicates that the following section is omitted except under certain conditions, or that the factors included in the following section depend upon certain conditions. In most cases, the description in the definition column is the same as the label output to the ss\_new files.

\subsection{Terminology for Fishing Mortality, F}
Here we introduce some terminology related to fishing mortality, $F$. This will provide context for some of the quantities that will be read from the starter file.

$f$ is fleet.

$t$ is a time step; continuous across years $y$ and seasons $s$; equivalent to year if only 1 season.

$a$ is age.

$s_{t,f,a}$ is age-specific selectivity for a fleet. If selectivity is length-specific, then age-specific selectivity due to length-selectivity is calculated as the dot product across length bins of length selectivity and the normal (or lognormal) distribution of length-at-age. If selectivity is both length- and age-based, which is an entirely normal concept in SS3, then age selectivity due to length selectivity is calculated first, then multiplied by the direct age selectivity. This compound age selectivity is used in the mortality calculations and is reported as $Asel2$ in report:32 of report.sso. See appendix to \citet{methotstock2013} for more detail on this. Selectivity can be sex-specific, and different growth morphs and platoons can have different age-selectivity due to the effect of length-selectivity on their unique size-at-age. This added dimension, $g$, for biological group is not included in the nomenclature here but exists in all the SS3 calculations.

$F_{t,f,a}$ is fishing mortality at age for fleet f.  There is no subscript for area because each fleet is defined to operate in only one area.

$F_{t,f}'$ is a fleet's fishing mortality for the age that has selectivity equal to 1.0. This is also termed F\_scalar or full_F in the SS3 system. If your model is using parameters for $F$, then the parameter values are for the $F'$. Note that some selectivity curves, like double normal, are explicit about having a maximum of 1.0. But other curves like logistic and combinations of length-selectivity and growth, may produce an age-selectivity curve that never reaches 1.0 and time-varying non-parametric selectivity will produce values >1.0 routinely. In all cases, the resultant $F_{t,f,a}$ comes from $F_{t,f}' * s_{t,f,a}$, so the range of the $F'$ compensates for the scale of the $s$.

Apical selectivity is the maximum age-specific selectivity and is not explicit in any internal calculation in SS3, it is just for reporting. If selectivity has a maximum value of 1.0, then apical\_F and F\_scalar are identical.

Fully-selected age range is not explicitly used in SS3, especially because SS3 applications routinely have multiple fleets with different selectivity patterns that may have little overlap.

$F$bar is the average $F$ over a user specified range of ages, implicitly the fully-selected range for the total $F$ from all the fleets.  Some SS3 output options will display $F$bar.

Annual_F is essentially the same as $F$bar and is an output quantity.

$F\_std$ is an output quantity that may be based on Annual_F or other calculated quantities like exploitation rate.  Importantly, the output values of $F\_std$ may be presented as a ratio relative to an equivalent benchmark (reference point) quantity; e.g. $F / F\_MSY$.  Further, the variance of $F\_std$ will be calculated and output.

$C_{t,f}$ is fleet-specific catch in a time step.

$B_{t,f}$ is fleet specific available biomass, e.g., total biomass filtered by fleet-specific age selectivity, $s_{t,f,a}$. Note that this is not adjusted by the $max(s_{t,f,a})$.


{
\setlength\extrarowheight{4pt}
\begin{landscape}
\subsection{Starter File Options (starter.ss)}	
%\centerline{\large{STARTER.SS}} 
%\vspace{0.1in}
%{\renewcommand{\arraystretch}{1.1}

\begin{longtable}{p{1.5cm} p{7.2cm} p{12.3cm}} 

 \hline
 \textbf{Value} & \textbf{Options} & \textbf{Description} \TBstrut\\ 
 \hline
 \endfirsthead
 
 \hline
 \textbf{Value} & \textbf{Options} & \textbf{Description} \TBstrut\\ 
 \hline
 \endhead
 
 \hline
 \endfoot
 
 \hline
 \multicolumn{3}{c}{\textbf{End of Starter File}} \Tstrut\Bstrut\\
 \hline
 \endlastfoot

 \#C this is a starter comment & Must begin with \#C then rest of the line is free form & All lines in this file beginning with \#C will be retained and written to the top of several output files \Tstrut\\
		
 \hline
 data\_ file.dat &  & File name of the data file \Tstrut\\
		
 \hline
 control\_ file.ctl &  & File name of the control file \Tstrut\\
   
 \hline		
 0 & Initial Parameter Values: & \multirow{1}{1cm}[-0.25cm]{\parbox{12.5cm}{Do not set equal to 1 if there have been any changes to the control file that would alter the number or order of parameters stored in the ss3.par file. Values in ss3.par can be edited, carefully. Do not run ss\_trans.exe from a ss3.par from v.3.24.}}\Tstrut\\
 & 0 = use values in control file; and&  \\
 & 1 = use ss3.par after reading setup in the control file. & \\
		
 \hline
 1 & Run display detail: &  \multirow{1}{1cm}[-0.25cm]{\parbox{12.5cm}{With option 2, the display shows value of each -logL component for each iteration and it displays where crash penalties are created}} \Tstrut\\
   & 0 = none other than ADMB outputs; & \\
   & 1 = one brief line of display for each iteration; and & \\
   & 2 = fuller display per iteration. & \\
		  
 \hline
 1 & Detailed age-structure report: & \multirow{1}{1cm}[-0.15cm]{\parbox{12.5cm}{Option 0 will forgo the writing of the Report file, but the ss\_summary file will be written that has minimal derived and estimated quantities. This is a useful option for some data-limited assessment approaches (e.g., XSSS or SSS). Option 1 will write out the full Report file. Option 2 will write out select items in the Report file and will omit some more detailed sections (e.g., numbers-at-age).}} \Tstrut\\
   & 0 = minimal output for data-limited methods; & \\
   & 1 = include all output (with wtatage.ss\_new); &  \\
   & 2 = brief output, no growth;  and &  \\	
   & 3 = custom output & \\
 \pagebreak
 
 \multicolumn{2}{l}{COND: Detailed age-structure report = 3} & \multirow{1}{1cm}[-0.25cm]{\parbox{12.5cm}{Custom report options: First value: -100 start with minimal items or -101 start with all items; Next Values: A list of items to add or remove where negative number items are removed and positive number items added, -999 to end. The \hyperlink{custom}{reporting numbers} for each item that can be selected or omitted are shown in the Report file next to each section key word.}} \Tstrut\\
 \multicolumn{1}{r}{-100} & & \\
 \multicolumn{1}{r}{  -5} & & \\
 \multicolumn{1}{r}{   9} & & \\
 \multicolumn{1}{r}{  11} & & \\
 \multicolumn{1}{r}{  15} & & \\
 \multicolumn{1}{r}{-999} & & \Bstrut\\
		 
 \hline
 0 & Write 1st iteration details: & \multirow{1}{1cm}[-0.25cm]{\parbox{12.5cm}{This output is largely unformatted and undocumented and is mostly used by the developer.}} \Tstrut\\
   & 0 = omit; and & \\
   & 1 = write detailed intermediate calculations to echoinput.sso during first call. & \Bstrut\\

 \hline
 0 & Parameter Trace: & \multirow{1}{1cm}[-0.25cm]{\parbox{12.5cm}{This controls the output to parmtrace.sso. The contents of this output can be used to determine which values are changing when a model approaches a crash condition.  It also can be used to investigate patterns of parameter changes as model convergence slowly moves along a ridge. In order to access parameter gradients option 4 should be selected which will write the gradient of each parameter with respect to each likelihood component}} \Tstrut\\
   & 0 = omit; & \\
   & 1 = write good iteration and active parameters; & \\
   & 2 = write good iterations and all parameters; & \\
   & 3 = write every iteration and all parameters; and & \\
   & 4 = write every iteration and active parameters. & \Bstrut\\
 \hline
 
 \pagebreak
 1 & Cumulative Report: & \multirow{1}{1cm}[-0.25cm]{\parbox{12.5cm}{Controls reporting to the file Cumreport.sso. This cumulative report is most useful when accumulating summary information from likelihood profiles or when simply accumulating a record of all model runs within the current subdirectory}} \Tstrut\\
   & 0 = omit;  & \\
   & 1 = brief; and & \\
   & 2 = full. & \\
	 
 \hline
 1 & Full Priors: & \multirow{1}{1cm}[-0.25cm]{\parbox{12.5cm}{Turning this option on (1) adds the log likelihood contribution from all prior values for fixed and estimated parameters to the total negative log likelihood. With this option off (0), the total negative log likelihood will include the log likelihood for priors for only estimated parameters.}} \Tstrut\\
   & 0 = only calculate priors for active parameters; and &	\\
   & 1 = calculate priors for all parameters that have a defined prior. & \\
	     
 \hline
 1 & Soft Bounds: & \multirow{1}{1cm}[-0.25cm]{\parbox{12.5cm}{This option creates a weak symmetric beta penalty for the selectivity parameters. This becomes important when estimating selectivity functions in which the values of some parameters cause other parameters to have negligible gradients, or when bounds have been set too widely such that a parameter drifts into a region in which it has negligible gradient. The soft bound creates a weak penalty to move parameters away from the bounds.}}\Tstrut\Bstrut\\
   & 0 = omit; and & \\
   & 1 = use. & \\
   & & \\
   & & \\
   & & \\
   & & \\

 \pagebreak
%  \hline
 1 & Number of Data Files to Output: & \multirow{1}{1cm}[-0.25cm]{\parbox{12.5cm}{All output files are sequentially output to data\_echo.ss\_new and need to be parsed by the user into separate data files. The output of the input data file makes no changes, retaining the order of the original file. Output files 2-N contain only observations that have not been excluded through use of the negative year denotation, and the order of these output observations is as processed by the model. At this time, the tag recapture data is not output to data\_echo.ss\_new. As of v.3.30.19, the output file names have changed; now a separate file is created for the echoed data (data\_echo.ss\_new), the expected data values given the model fit (data\_expval.ss), and any requested bootstrap data files (data\_boot\_x.ss where x is the bootstrap number). In versions before v.3.30.19, each of these outputs was printed to a single file called data.ss\_new.}} \Tstrut\Bstrut\\
   & 0 = none; As of v.3.30.16, none of the .ss\_new files will be produced;& \Bstrut\\
   & 1 = output an annotated replicate of the input data file; & \Tstrut\Bstrut\\
   & 2 = add a second data file containing the model's expected values with no added error. ; and & \Tstrut\Bstrut\\
   & 3+ = add N-2 parametric bootstrap data files. & \Tstrut\\
   & & \Bstrut\\
  %  & & \\

 \hline
 %\pagebreak
 8 & Turn off estimation: &  \multirow{1}{1cm}[-0.25cm]{\parbox{12.5cm}{The 0 option is useful for (-1) quickly reading in a messy set of input files and producing the annotated control.ss\_new and data\_echo.ss\_new files, or (0) examining model output based solely on input parameter values. Similarly, the value option allows examination of model output after completing a specified phase. Also see usage note for restarting from a specified phase.}} \Tstrut\\
   & -1 = exit after reading input files; & \\
   & 0 = exit after one call to the calculation routines and production of sso and ss\_new files; and & \\
   & <positive value> = exit after completing this phase. & \Bstrut\\	  
	     
 \hline
 1000 & MCMC burn interval & Number of iterations to discard at the start of an MCMC run. \Tstrut\Bstrut\\
	   
 \hline
 %\pagebreak
 200 & MCMC thin interval & Number of iterations to remove between the main period of the MCMC run. \Tstrut\\
	
 \pagebreak
%  \hline 
 0.0 & \hyperlink{Jitter}{Jitter:} & \multirow{1}{1cm}[-0.25cm]{\parbox{12.5cm}{The jitter function has been revised with v.3.30. Starting values are now jittered based on a normal distribution with the pr(P\textsubscript{MIN}) = 0.1\% and the pr(P\textsubscript{MAX}) = 99.9\%. A positive value here will add a small random jitter to the initial parameter values. When using the jitter option, care should be given when defining the low and high bounds for parameter values and particularly -999 or 999 should not be used to define bounds for estimated parameters.}} \Tstrut\\ 
	 & 0 = no jitter done to starting values; and & \\
	 & >0 starting values will vary with larger jitter values resulting in larger changes from the parameter values in the control or par file. & \\
	 & & \\
	
 \hline
 -1 & SD Report Start: & \Tstrut\\
    & -1 = begin annual SD report in start year; and & \\
    & <year> = begin SD report this year. & \Bstrut\\
	      
 \hline
%  \pagebreak
 -1 & SD Report End: & \Tstrut\\
    & -1 = end annual SD report in end year; & \\
    & -2 = end annual SD report in last forecast year; and & \\
    & <value> = end SD report in this year. & \Bstrut\\
	   
 \hline
 2 & Extra SD Report Years: & \multirow{1}{1cm}[-0.25cm]{\parbox{12.5cm}{In a long time series application, the model variance calculations will be smaller and faster if not all years are included in the SD reporting. For example, the annual SD reporting could start in 1960 and the extra option could select reporting in each decade before then.}} \Tstrut \Bstrut\\
   & 0 = none; and & \\
   & <value> = number of years to read. & \Bstrut\\
  %  & & \\

 %\pagebreak 
 \hline  
 \multicolumn{3}{l}{COND: If Extra SD report years > 0} \Tstrut\\

 %\pagebreak
 %\hline
 \multicolumn{1}{r}{1940 1950} & \multirow{1}{1cm}[-0.25cm]{\parbox{19.5cm}{Vector of years for additional SD reporting. The number of years need to equal the value specified in the above line (Extra SD Report Years).}} \\
  & & \\
 
 \hline
 0.0001 & Final convergence & \multirow{1}{1cm}[-0.25cm]{\parbox{12.5cm}{This is a reasonable default value for the change in log likelihood denoting convergence. For applications with much data and thus a large total log likelihood value, a larger convergence criterion may still provide acceptable convergence.}} \Tstrut\Bstrut\\
   & & \Bstrut\\
   & & \Bstrut\\
	%  & & \\ 
 
 \hline
 0 & Retrospective year: & \multirow{1}{1cm}[-0.25cm]{\parbox{12.5cm}{Adjusts the model end year and disregards data after this year. May not handle time varying parameters completely.}} \Tstrut\\
   & 0 = none; and & \\
   & -x = retrospective year relative to end year. & \Bstrut\\
  
 \hline
 0 & Summary biomass min age & \multirow{1}{1cm}[-0.25cm]{\parbox{12.5cm}{Minimum integer age for inclusion in the summary biomass used for reporting and for calculation of total exploitation rate.}} \Tstrut\\
   & & \\ 

 \hline
%  \pagebreak
 1 & Depletion basis: & \multirow{1}{1cm}[-0.25cm]{\parbox{12.5cm}{Selects the basis for the denominator when calculating degree of depletion in SB. The calculated values are reported to the SD report.}} \Tstrut\\
   & 0 = skip; & \\
   & 1 = X*SB0; & Relative to virgin spawning biomass. \\
   & 2 = X*SB\textsubscript{MSY}; & Relative to spawning biomass that achieves MSY. \\
   & 3 = X*SB\textsubscript{styr}; and & Relative to model start year spawning biomass. \\
   & 4 = X*SB\textsubscript{endyr}. & Relative to spawning biomass in the model end year. \\
   & 5 = X*Dynamic SB0 & Relative to the calculated dynamic SB0. \\
   & & Use hundreds and tens place to invoke multi-year trailing average, the ones place for depletion basis value, and the decimal place invoke log(ratio). This approach for multi-year log(ratio) was implemented with v.3.30.24 (2024). \\
   & & Example: 122.1 invokes a 12 year trailing average for X*SB\textsubscript{MSY} using log(ratio). \\
   & & Note that the trailing average always goes up to the endyr and once it gets to the first forecast year it stops the trailing average. \Bstrut\\
  
 \hline
 1 & Fraction (X) for depletion denominator & Value for use in the calculation of the ratio for SB\textsubscript{y}/(X*SB0). \Tstrut\Bstrut\\

%  \hline
 \pagebreak
 1 & SPR report scaling: & \multirow{1}{1cm}[-0.25cm]{\parbox{12.5cm}{SPR is the equilibrium SB per recruit that would result from the current year's F-at-age. The quantities identified by 1, 2, and 3 here are calculated in the benchmarks section. Then the one specified here is used as the selected }} \Tstrut\\
   & 0 = skip; & \\
   & 1 = use 1-SPR\textsubscript{target}; & \\
   & 2 = use 1-SPR at MSY; & \Tstrut\\
   & 3 = use 1-SPR at B\textsubscript{target}; and & \multirow{1}{1cm}[-0.25cm]{\parbox{12.5cm}{denominator in a ratio with the annual value of (1 - SPR). This ratio (and its variance) is reported to the SD report output for the years selected above in the SD report year selection.}} \Tstrut\\
   & 4 = no denominator, so report actual 1-SPR values. & \\
  
%  \pagebreak
\hline 
 4 & F reporting units (F_std): & \multirow{1}{1cm}[-0.25cm]{\parbox{12.5cm}{In addition to SPR, an additional proxy for annual F can be specified here. As with SPR, the selected quantity will be calculated annually and in the benchmarks section. The ratio of the annual value to the selected (see F report basis below) benchmark value is reported to the SD report vector. Options 1 and 2 use total catch for the year and summary abundance at the beginning of the year, so combines seasons and areas. But if most catch occurs in one area and there is little movement between areas, this ratio is not informative about the F in the area where the catch is occurring. Option 3 is a simple sum of the full F's by fleet, so may provide non-intuitive results when there are multi areas or seasons or when the selectivities by fleet do not have good overlap in age. Option 4 is a real annual F calculated as a numbers weighted F for a specified range of ages (read below). The F is calculated as Z-M where Z and M are each calculated an ln(N\textsubscript{t+1}/N\textsubscript{t}) with and without F active, respectively. The numbers are summed over all biology morphs and all areas for the beginning of the year, so subsumes any seasonal pattern.}} \Tstrut\Bstrut\\
   & 0 = skip; & \\
   & 1 = exploitation rate in biomass; & \\
   & 2 = exploitation rate in numbers; & \\
   & 3 = sum(apical F's by fleet); & \\
   & 4 = Fbar: numbers weighted F for range of ages; and & \\
   & 5 = Fbar: unweighted average F for range of ages. & \\
   & & \\
   & & \\
   & & \Bstrut\Bstrut\\
   & & \Bstrut\Bstrut\\
   & & \Bstrut\\ 
  %  & & \\ 
  
 \hline
 %\pagebreak
 \multicolumn{2}{l}{COND: If F std reporting $\geq$ 4} & \multirow{1}{1cm}[-0.25cm]{\parbox{12.5cm}{Specify range of ages. Upper age must be less than max age because of incomplete handling of the accumulator age for this calculation.}} \Tstrut\\
 \multicolumn{1}{r}{3 7}  & Age range if F std reporting = 4. & \Tstrut\Bstrut\\

%  \hline
 \pagebreak
 1 & F report basis: & \multirow{1}{1cm}[-0.25cm]{\parbox{12.5cm}{Selects the denominator to use when reporting the F std report values. A new option to allow for the calculation of a multi-year trailing average in F was implemented in v.3.30.16. This option is triggered by appending the number of years to calculate the average across where an input of 1 or 11 would result in the SPR\textsubscript{target} with no changes. Alternatively a value of 21 would calculate F as SPR\textsubscript{target} with a 2-year trailing average.}} \Tstrut\\
   & 0 = not relative, report raw values; & \\
   & 1 = use F std value relative to SPR\textsubscript{target}; & \\
   & 2 = use F std value relative to F\textsubscript{MSY}; and & \\
   & 3 = use F std value relative to F\textsubscript{Btarget}. & \\
   & use tens digit (1-9) to invoke multi-year (up to 9 yrs) F std & \\
   & use 1 as hundreds digit to invoke log(ratio) & \Bstrut\\

  \hline
  %\pagebreak
  0.01 & MCMC output detail: & \multirow{1}{1cm}[-0.25cm]{\parbox{12.5cm}{Specify format of MCMC output. This input requires the specification of two items; the output detail and a bump value to be added to the ln(R0) in the first call to MCMC. A bias adjustment of 1.0 is applied to recruitment deviations in the MCMC phase, which could result in reduced recruitment estimates relative to the MLE when a lower bias adjustment value is applied. A small value, called the ``bump'', is added to the ln(R0) for the first call to MCMC in order to prevent the stock from hitting the lower bounds when switching from MLE to MCMC. If you wanted to select the default output option and apply a bump value of 0.01 this is specified by 0.01 where the integer value represents the output detail and the decimal is the bump value.}} \Tstrut\Bstrut\\
  & 0 = default; & \\
  & 1 = output likelihood components and associated lambda values; & \\
  & 2 = write report for each mceval; and & \\		 
  & 3 = make output subdirectory for each MCMC vector. & \Bstrut\\
  & & \Tstrut\Bstrut\\
  & & \\ 
  & & \\ 
  % & & \\  		 
  
  \hline
  \hypertarget{ALK}{0} & Age-length-key (ALK) tolerance level & This effect is disabled in code, enter 0. \Tstrut\Bstrut\\
  % \multirow{1}{1cm}[-0.25cm]{\parbox{12.5cm}{Value of 0 will not apply any compression.  Values > 0 (e.g., 0.0001) will apply compression to the ALK which will increase the speed of calculations.  The size of this value will impact the run time of your model, but one should be careful to ensure that the value used does not appreciably impact the estimated quantities relative to no compression of the ALK.  The suggested value if applied is 0.0001.}} \Tstrut\Bstrut\\ 
  % & & \\
  % & & \Tstrut\\
  % & & \Tstrut\Bstrut\\

  \pagebreak
  % \hline  
  \multicolumn{2}{l}{COND: Seed Value (i.e., 1234)}& \multirow{1}{1cm}[-0.25cm]{\parbox{12.5cm}{Specify a seed for data generation. This feature is not available in versions prior to v.3.30.15 This is an optional input value allowing for the specification of a random number seed value. If you do not want to specify a seed, skip this input line and end the starter file with the check value (3.30).}} \Tstrut\Bstrut\\
  & & \\ 
  & & \Bstrut\\
  & & \\
  
%  \pagebreak
 \hline
 \hypertarget{Convert}{3.30} & Model version check value. & \multirow{1}{1cm}[-0.25cm]{\parbox{12.5cm}{A value of 3.30 indicates that the control and data files are currently in v.3.30 format. A value of 999 indicates that the control and data files are in a previous v.3.24 version. The ss\_trans.exe executable should be used and will convert the v.3.24 files the control.ss\_new and data\_echo.ss\_new files to the new format. All ss\_new files are in the v.3.30 format, so starter.ss\_new has v.3.30 on the last line. The mortality-growth parameter section has a new sequence and v.3.30 cannot read a ss3.par file produced by v.3.24 and earlier, so ensure that read par file option at the top of the starter file is set to 0. The \hyperlink{ConvIssues}{Converting Files from Stock Synthesis v.3.24} section has additional information on model features that may impede file conversion.}} \Tstrut\Bstrut\\
     & & \\  
     & & \\  
	   & & \\
     & & \\
   	 & & \\
     & & \\  
     & & \\  
     & & \\

\end{longtable}
\end{landscape}
}
\restoregeometry





\pagebreak
