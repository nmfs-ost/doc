\hypertarget{AdvancedSettings}{}
\section[Advanced Stock Synthesis Configuration Settings and Advice]{\protect\hyperlink{AdvancedSettings}{Advanced Stock Synthesis Configuration Settings and Advice}}

\hypertarget{TVpara}{}
\subsection[Using Time-Varying Parameters]{\protect\hyperlink{TVpara}{Using Time-Varying Parameters}}

\hypertarget{tvOrder}{}
\subsubsection[Time-Varying Parameters]{\protect\hyperlink{tvOrder}{Time-Varying Parameters}}

Starting in v.3.30, mortality-growth, some stock-recruitment, catchability, and selectivity base parameters can be time varying. Note that as of v.3.30.16, time-varying parameters cannot be used with tagging parameters. There are four ways a parameter can be time-varying in SS3:
\begin{enumerate}
    \item Environmental or Density dependent Linkages: Links the base parameter with environmental data or a model derived quantity.
	\item Parameter deviations: Creates annual deviations from the base parameter during a user-specified range of years.
	\item Time blocks: The base parameter is changed during a ``block'' (or ``blocks'') of time (i.e., one or more consecutive years) as specified by the user.
	\item Trends: A trend (shape: cumulative normal distribution function) is applied to the parameter. Trends are specified using the same input column as time blocks, but with different codes. This means that trends and time blocks cannot be used simultaneously for the same base parameter.
\end{enumerate}

Environmental and density dependent linkages, parameter deviations, and either time blocks or trends can be applied to the same base parameter. The model processes each time-varying parameter specification (first time blocks and trends, then environmental linkages, then parameter deviations) and creates a time-series of intermediate values that are used as the model subsequently loops through years.

\begin{figure}[ht]
	\begin{center}
		\includegraphics[alt={Some examples of time-varying setups.},scale = 0.4]{TimeVarying}\\
	\end{center}
	\caption{Some examples of time-varying setups.}
	\label{(fig:timevarying)}
\end{figure}

\pagebreak

\subsubsection{Specification of Time-Varying Parameters: Long Parameter Lines} 

Time-varying specifications for a parameter are invoked using elements 8 - 14 in the \hyperlink{paraOrder}{long parameter line setup}. Each element and the options for selection related to time-varying parameters are as described below.

\hypertarget{EnvVar}{}
\begin{itemize}

\item Environmental or Density Dependent Link and variable (env\_var\&link; element 8)

	\begin{itemize}
	   \item The environmental or density dependent link and variable input is two inputs specified using a single three-digit number. The hundreds place contains the option for the link function, while the tens and ones place is used to specify the environmental variable or derived quantity to which the parameter is linked. Note that environmental variables can only be included on an annual basis, so seasonal models would have the same effect applied to all seasons. If the environmental link and variable input is positive, then the parameter is linked to a variable specified in the data file environmental data; if it is negative, then the parameter is linked to a derived quantity. For example, env\_var\&link input 103 would use link type 1 and apply it to environmental data column 3, while the input -103  would use link type 1 and apply it to the ``-3'' column which is ln(relative summary biomass). The other options for both elements are enumerated below.
	   \item The link function options (hundreds place) for the env\_var\&link input are:
	   \begin{itemize}
	       \item 1 = exponential scalar: $P_{y} = P_{base}e^{P_{t}E_{y}}$
		   \item 2 = linear offset: $P_{y} = P_{base} + P_{t}E_{y}$
		   \item 3 = Bounded replacement: $P_{y} = min(P_{base})+\frac{max(P_{base})-min(P_{base})}{1+e^{P_tE_y+ln((P_{base}-min(P_{base})+0.0000001)/(max(P_{base})-P_{base}+0.0000001))}}$
		   \item 4 = Logistic: $P_{y} = P_{base}\frac{2}{1+e^{-P_{t2}(E_{y}-P_{t1})}}$
	   \end{itemize}
		where:
	   \begin{itemize}
	       \item $P_{y}$ = Parameter value in year $y$
           \item $P_{base}$ = Base parameter value
           \item $P_{t}$ = Link parameter value
           \item $P_{t1}$ = First of 2 link parameters (offset)
           \item $P_{t2}$ = Second of 2 link parameters (slope)
           \item $E_{y}$ = Environmental index value or derived quantity value in year $y$
           \item $min(P_{base})$ = the minimum parameter bound of base parameter
           \item $max(P_{base})$ = the maximum parameter bound of base parameter
        \end{itemize}
		\item The variable options (tens and ones place, or $E_{y}$) for the env\_var\&link input are either 1) a positive integer from 1 to 99 referencing a time-series located in the \hyperlink{env-dat}{environmental data section} of the data file, or 2) a negative value of -1 to -4 where $E_y$ is one of the following model-derived quantities:
		\begin{itemize}
			\item -1;  for ln(relative spawning biomass)
			\item -2;  for recruitment deviation
			\item -3;  for ln(relative summary biomass) (e.g., current year summary biomass divided by the unfished summary biomass)
			\item -4;  for ln(relative summary numbers)
		\end{itemize}
		\item The four derived quantities are all calculated at the beginning of each year within the model, so they are available to use as the basis for time-varying parameter links without violating any order of operations rules.
	\end{itemize}
	
\item Deviation Link (element 9). A positive integer invokes parameter deviations, but otherwise should be left as 0. SS3 expects the estimated deviations to be normal in distribution and the deviation values are multiplied by the standard error parameter as they are used. This differs from recruitment deviations and from the approach in SS3 v.3.24. Link options for parameter deviations are:
	\begin{itemize}
		\item 1 = multiplicative: $P_y = P_{base,y}e^{\text{dev}_y*\text{dev}_{se}}$,
		\item 2 = additive: $P_y = P_{base,y} + \text{dev}_y*\text{dev}_{se}$,
		\item 3 = random walk. Random walk options are implemented by using $\rho$ in the objective function. $P_y = P_{base,y} + \sum_{n=1}^{y} \text{dev}_n*\text{dev}_{se}$
		\item 4 = mean reverting random walk with $\rho$.
		\begin{itemize}
		    \item $X_1 = \text{dev}_1*\text{dev}_{se}$
			\item $P_1 = P_{base,y} + X_1$
			\item $X_y = \rho*X_{y-1} + \text{dev}_y*\text{dev}_{se}$
			\item $P_y = P_{base,y} + X_y$
		\end{itemize}
		\item 5 = mean reverting random walk with $\rho$ and a logit transformation to stay within the minimum and maximum parameter bounds (approach added in v.3.30.16)
	    \begin{itemize}
		    \item $X_1 = \text{dev}_1*\text{dev}_{se}$
			\item $R = P_{max} - P_{min}$
			\item $Y_1 = ln(\frac{P_{base,y} - P_{min} + \text{nil}}{P_{max} - P_{base,y} + nil})$
			\item $P_1 = P_{min} + \frac{R}{1 + e^{-Y_1 - X_1 }}$. For the first year.
			\item $X_y = \rho*X_{y-1} + \text{dev}_y*\text{dev}_{se}$ 
			\item $Y_y = ln(\frac{P_{base,y} - P_{min} + nil}{P_{max} + P_{base,y} + nil})$
			\item $P_1 = P_{min} + \frac{R}{1 + e^{-Y_y - X_y }}$. For years after the first year.
		\end{itemize}
		\item 6 = mean reverting random walk with penalty to keep the root mean squared error (RMSE) near 1.0. Same as case 4, but with penalty applied.
		\item The option of extending the final model year deviation value subsequent years (i.e., into the forecast period) was added in v.3.30.13. This new option is specified by selecting the appropriate deviation link option and appending a 2 at the front (e.g., 25), which will use the final year deviation value for all forecast years.
	\end{itemize}
	where: 
	\begin{itemize}
	     \item $P_{y}$ = Parameter value in year $y$
         \item $P_{base,y}$ = Base parameter value for year $y$
		 \item $\text{dev}_y$ = deviation in year $y$
		 \item $\text{dev}_{se}$ = standard error of the deviation
		 \item nil is a small value (e.g., 0.0000001)
	\end{itemize}
\item Deviation Minimum Year (element 10). Year deviations start for the parameter. This must be specified if using parameter deviations, but otherwise should be left as 0.
	
\item Deviation  Maximum Year (element 11). Year deviations end for parameter. This must be specified if using parameter deviations, but otherwise should be left as 0.
	
\item Deviation Phase (element 12). The phase in which the deviations for the parameter should be estimated. This must be specified if using parameter deviations, but otherwise should be left as 0.
%is there a recommended phase to use if wanting to estimate devs?
	
\item Use Time Blocks or Trends (element 13). Time blocks and trends are both specified using this input. If neither are used, this should be left as 0. For trend options, the cumulative normal distribution function is used as the shape of the trend in all cases, but the parameterization differs. In general, the trend used is: 
    \begin{equation}
	    P_y = P_{base} + P_{\text{offset}}\phi(\frac{y - P_{\text{infl}}}{P_{width}})
	\end{equation}
	where
	\begin{itemize} 
	    \item $P_y$ is the final parameter value in year $y$
		\item $P_{base}$ is the base parameter value
		\item $P_{\text{offset}}$ is the parameter offset value
		\item $\phi$ is the standard cumulative normal distribution function
		\item $P_{\text{infl}}$ is the inflection year (i.e., the year in which half of the total change from the base parameter has occurred)
		\item $P_{width}$ is the standard deviation.
	\end{itemize}
In all cases, 3 parameters are estimated and hence 3 short parameter lines are required. These parameter lines differ amongst the trend options.

The input value options for element 13 are:
	\begin{itemize}
% TODO: add more information about how the above equation relates to theoptions -1 and -3.
		\item >0: time block index for parameter. See the \hyperlink{timeblocks}{time blocks section} of the control file for more information on specifying time blocks.
		\item -1: Trend Offset option. Three parameters are estimated: end trend value as a logistic offset (input as $ln(P_{\text{offset}})$), inflection year logistic offset (input as $ln(P_{\text{infl}})$), and width ($P_{width}$). Offset trend value is in natural log space. Inflection year is also in natural log space and offset from ln(0.5). Width is directly specified.
		\item -2: Trend Direct input option. In this case, $P_{\text{offset}} = 1$. Three parameters are input via short parameter lines: end trend parameter value ($P_y$ where $y$ is the final year), inflection year ($P_{\text{infl}}$), and width ($P_{width}$). 
		\item -3: Trend Fractional option. In this case, $P_{\text{offset}} = 1$. Three parameters will be estimated: end trend parameter value as a fraction of base parameter maximum - minimum, inflection year as a fraction of end year - start year, and width ($P_{width}$). Width is directly input.
	\end{itemize}
	
\item Time Block Functional Form (element 14). Leave as 0, unless time blocks are used.
	\begin{itemize}
		\item 0: multiplicative parameter ($P_{block} = P_{base}*e^{P_t}$)
		\item 1: additive parameter ($P_{block} = P_{base} + P_t$)
		\item 2: replace parameter ($P_{block} = P_t$)
		\item 3: random walk across blocks ($P_{block} = P_{block,-1} + P_t$)
	\end{itemize}
	where:
	\begin{itemize}
        \item $P_{block}$ = Final parameter value in time block $block$
        \item $P_{base}$ = Base parameter value
		\item $P_{t}$ = Time-varying parameter value for a time block
		\item $P_{block,-1}$ = Final parameter value in the previous time block
     \end{itemize}
\end{itemize}


Code for the deviation link can be found in \href{https://github.com/nmfs-ost/ss3-source-code/blob/main/SS_timevaryparm.tpl}{\verb|SS\_timevaryparm.tpl|}, search for ``SS\_Label\_Info\_14.3''.


\subsubsection{Specification of Time-Varying Parameters: Short Parameter Lines} 

If a time-varying specification set up in the long parameter lines for a particular section requires additional parameters, short parameter lines need to be created following the long parameter lines for the section (unless \hyperlink{autogen}{autogeneration} is used, which creates short parameter lines in \verb|control.ss\_new| upon running the model). The number of parameter lines required depends on the time-varying parameter specification.

For example, if two parameters were specified to have environmental linkages in the MG parameter section, below the MG parameters would be two parameter lines (when not auto-generating these lines), which is an environmental linkage parameter for each time-varying base parameter:

\begin{longtable}{p{0.7cm} p{0.7cm} p{0.7cm} p{1cm} p{1.4cm} p{1cm} p{1cm} p{6.7cm}}
	\hline
	&    &      & Prior &  Prior & Prior &  & \Tstrut\\
	LO & HI & INIT & Value &  SD    & Type  & Phase & Parameter Label \Bstrut\\
	\hline
	\endfirsthead
	
	\hline
	&    &      & Prior &  Prior & Prior &  & \Tstrut\\
	LO & HI & INIT & Value &  SD    & Type  & Phase & Parameter Label \Bstrut\\
	\hline
	\endhead
	
	\endfoot
	
	\endlastfoot
	
	\multicolumn{7}{l}{COND: Only if MG parameters are time-varying} \Tstrut\\
	-99 & 99 & 1 & 0 & 0.01 & 0 & -1 &\#Wtlen\_1\_Fem\_ENV\_add \Tstrut\\
	-99 & 99 & 1 & 0 & 0.01 & 0 & -1 &\#Wtlen\_2\_Fem\_ENV\_add \Bstrut\\
	\hline
\end{longtable}

In Stock Synthesis v.3.30, the time-varying input short parameter lines are organized such that all parameters that affect a base parameter are clustered together with time blocks (or trend) first, then environmental linkages, then parameter deviations. For example, if the mortality-growth (MG) base parameters 3 and 7 had time varying changes, the order would look like:
 
 \begin{center}
 	\begin{longtable}{p{5cm} p{10cm}}
 		\hline
 		MG base parameter 3 & Block parameter 3-1 \Tstrut\\
 		& Block parameter 3-2 \\
 		& Environmental link parameter 3-1 \\
 		& Deviation se parameter 3 \\
 		& Deviation $\rho$ parameter 3 \Bstrut\\
 		MG base parameter 7 & Block parameter 7-1 \\
 		& Deviation se parameter 7 \\
 		& Deviation $\rho$ parameter 7 \Bstrut\\
 		\hline	 	                    		
 	\end{longtable}
 \end{center}
 
The number of short parameter lines for each time-varying setup selected depends on the selection options. The \hyperlink{autogen}{autogeneration} feature can be used to figure out which parameter lines are needed. The short parameter lines needed for different time-varying options are:
\begin{itemize}
	\item Environmental Linkages: Requires 1 short parameter line ($P_{t}$), except for link option 4, which requires 2 short parameter lines ($P_{t1}$ and $P_{t2}$).
	\item Parameter deviations: Requires 2 short parameter lines, one for the standard error ($\text{dev}_{se}$), followed by one for $\rho$. Note that an input for $\rho$ is required but only used with random walk options. For the random walk options, $\rho$ can be set at 1 for a random walk with no drift or >1 for a random walk with drift.
	\item Time Blocks: One parameter for each time block ($P_{t}$) set up in the pattern.
	\item Trends: Requires 3 parameter lines. The interpretation of the parameters differs by the trend option selected, but in general they are a parameter for the final parameter value, a parameter for the inflection point year, and a parameter for the width (i.e., the standard deviation).
\end{itemize}

\subsubsection{Example Time-varying Parameter Setups}

The time-varying parameter options in Stock Synthesis are flexible. Below are some example setups that illustrate how the time-varying options could be used in a model, although there are many more possible setups.

\myparagraph{Environmental and density dependent linkages}

\begin{itemize}
	\item Suppose growth rate is found to be linked with an index of water temperature. The water temperature proxy could be input into the data file as environmental data. If it is input as index number 1, the growth parameter $K$ (if using a von Bertalanffy growth equation) could be linked to the water temperature proxy data by specifying the code ``201'' in the env\_var\&link function input. This would establish an offset link between the parameter and the temperature proxy. One additional parameter line is required after the ``MG parameter'' long parameter lines section.
	\item Suppose for a fishery, selectivity is thought to shift depending on population size. Smaller fish are selected when there are lower population numbers, while larger fish are selected when there are higher population numbers. The selectivity parameter could be made time-varying using the code ``-104'' in the env\_var\&link option, which assumes an exponential scalar link between the base selectivity parameter and the time varying parameter value. One additional parameter line is required at the end of the selectivity long parameter lines section.
\end{itemize}

\myparagraph{Parameter Deviations}

\begin{itemize}
	\item Suppose a selectivity parameter is thought to drift every year during 2000-2010. This could be represented using a random walk link option available within the parameter deviations options. To implement this, the user could input 3 into the ``dev link'' input on the long parameter line for the selectivity parameter, and then input values 2000 and 2010 for ``dev min yr'' and ``dev max yr'', respectively. The dev phase could be set to 3. With this setup, 2 additional short parameter lines would be expected, one for the standard error and one for $\rho$. Both of these will be used since a random walk option is selected. To use a random walk without drift, $\rho$ is set at 1 with a negative phase.
\end{itemize}

\myparagraph{Time Blocks}

\begin{itemize}
	\item Offset approach: One or more time blocks are created and cover all or a subset of the years. Each block gets a parameter that is used as an offset from the base parameter (time block functional form 1). In this situation, typically the base parameter and each of the offset parameters are estimated. In years not covered by blocks, the base parameter alone is used. However, if blocks cover all the years, then the value of the block parameter is completely correlated with the mean of the block offsets, so model convergence and variance estimation could be affected.  The recommended approach when using offsets is to not have all years covered by blocks or to fix the base parameter value at a reasonable level when doing offsets for all years.	
	
	\item Replacement approach, Option A: Time blocks are created which cover a subset of the years. The base parameter is used in the non-block years and the value of the base parameter is replaced by the block parameter in each respective block (time block functional form 2). In this situation, typically the base parameter and each of the block parameters are estimated.	
	
	\item Replacement approach, Option B: Replacement time blocks are created for all the years, so the base parameter is simply a placeholder that is always replaced by a block parameter (time block functional form 2). In this situation, do not allow the model to estimate the base parameter and only estimate the corresponding block replacement parameters, otherwise, the search algorithm will be attempting to estimate parameters that do not contribute to the log likelihood, so model convergence and variance estimation could be affected.
\end{itemize}

\myparagraph{Trends}

\begin{itemize}
	\item Suppose natural mortality was thought to increase from 0.1 to 0.2 during 2000 to 2010. This could be input as a trend. First, the natural mortality parameter would be fixed at an initial value of 0.1. Then, a value of -2 could be input into the ``use block'' column of the natural mortality long parameter line to indicate that the direct input option for trends should be used. The long parameter line for M could look like:
	\begin{center}
		\begin{longtable}{p{1cm} p{1cm} p{1cm} p{1.5cm} p{1cm} p{1.5cm} p{1.5cm} p{1.5cm} p{3cm}}
			
			\hline
			LO \Tstrut & HI & INIT & <other entries> & PHASE & <other entries> & Use\_Block & Block Fxn & Parameter Label \Bstrut\\
			\hline
			0          & 4 & 0.1 & \multicolumn{1}{c}{...} & -1 & \multicolumn{1}{c}{...} & -2 & 0 & \#M \Bstrut\\
			\hline
		\end{longtable}
	\end{center}

	\item Three short parameter lines are then expected after the mortality-growth long parameter lines, one for the final value, one for the inflection year and one for the width. The final value could be fixed by using 0.2 as the final value on the short parameter line and a negative phase value. The inflection year could be fixed at 2005 by inputting 2005 for the inflection year in the short parameter line with a negative phase. Finally, the width value (i.e., standard deviation of the cumulative normal distribution) could be set at 3 years. The short parameter lines could look like:
	
	\begin{longtable}{p{0.7cm} p{0.7cm} p{0.7cm} p{1cm} p{1.4cm} p{1cm} p{1cm} p{6.7cm}}
	\hline
	&  &      & Prior &  Prior & Prior &  & \Tstrut\\
	LO & HI & INIT & Value &  SD    & Type  & Phase & Parameter Label \Bstrut\\
	\hline
	\endfirsthead
	
	\hline
	&  &      & Prior &  Prior & Prior &  & \Tstrut\\
	LO & HI & INIT & Value &  SD    & Type  & Phase & Parameter Label \Bstrut\\
	\hline
	\endhead
	
	\endfoot
	
	\endlastfoot
	
	0.001 & 4    & 0.2  & 0 & 0.01 & 0 & -1 & \#M\_TrendFinal \Tstrut\\
	1999  & 2011 & 2005 & 0 & 0.01 & 0 & -1 & \#M\_TrendInfl \Bstrut\\
	-99   & 99   & 3    & 0 & 0.01 & 0 & -1 & \#M\_TrendWidth\_yrs \Bstrut\\
	\hline
\end{longtable}
\end{itemize}



\hypertarget{tvgrowth}{}
\subsubsection[Time-Varying Growth Considerations]{\protect\hyperlink{tvgrowth}{Time-Varying Growth Considerations}}
When time-varying growth is used, there are some additional considerations to be aware of:
\begin{itemize}
	\item Growth in the forecast with time blocks: Growth deviations propagate into the forecast because growth is by cohort according to the current year's growth parameters. The user can select which growth parameters get used during the forecast by setting the end year of the last block, if using time blocks. If the last block ends in the model's end year, then the growth parameters in effect during the forecast will be the base parameters. By setting the end year of the last block to one year past the model end year (endyr), the model will continue the last block's growth parameter levels throughout the forecast.
	\item The equilibrium benchmark quantities (MSY, F40\%, etc.) previously used the model end year's (endyr) body size-at-age, which is not in equilibrium. Through the forecast file, it is possible to specify a range of years over which to average the size-at-age used in the benchmark calculations. An option to create equilibrium growth from averaged growth parameters would be a more realistic option and is under consideration, but is not yet available.
	% Which input in forecast?? The benchmark years input? I couldn't find this option...
	% Details about a potentially better solution.
	%\item The mean length of fish in the plus group in the start year takes into account the infinity tail of numbers of fish at age and the continued growth of these tail fish out to 3 times the age of the plus group. This has nil effect if fish have already reached Linfinity, but can be noticeable if they have not. This detail is displayed only in echoinput.sso.  Then in each year starting with the first year with time-varying growth parameters, the size in the plus group is updated according to the weighted mean of fish already there and fish just entering the plus group.  If fish have not yet reached Linfinity by the time they reach the plus group, then you will see a decline over time in mean length of fish in the plus group simply because the numbers of fish in the plus group goes down over time. However, this is only done in years with time-varying growth. This is a topic for future Stock Synthesis evolution.
\end{itemize}

\hypertarget{tv-sr}{}
\subsubsection[Time-Varying Stock-Recruitment Considerations]{\protect\hyperlink{tv-sr}{Time-Varying Stock-Recruitment Considerations}}
\begin{itemize}
	
	\item The $\sigma_R$ and autocorrelation parameters cannot be time-varying. 
	% Is this true? Or can they be time varying but it is not advisable because it doesn't make much sense?
	
	\item The autocorrelation parameter cannot be estimated accurately within SS3 \citep{johnson-can-2016}, so external (i.e., external to SS3) estimation for selecting an autocorrelation value is currently recommended. The autocorrelation of the recruitment deviations appears in the report file, which can aid in selecting the autocorrelation value.
		
	\item The value of R0 and steepness in the initial year are used within virgin calculations and within the benchmarks for calculation of the denominator in depletion estimates. The average value of R0 and steepness in the range of years specified as the benchmark years inputs 9 and 10 (see \hyperlink{fore-specify}{the forecast file specifications}) is used for MSY-type calculations. 
	%So, for example, a long-term climate effect could cause R0 to change over time and B\textsubscript{MSY} could now be calculated for some future range of years.
	
	\item The spawner-recruit regime parameter is a modifier on R0. The regime shift parameter line allows for multi-year or environmentally driven deviations from R0 without changing R0 itself. The regime shift base parameter should have a base value of 0.0 and not be estimated (i.e., have a negative phase). Similar to the cohort-growth deviation, it serves simply as a base for adding time-varying adjustments.
	
	\item The same algebraic effect on the calculated recruitment can be achieved by different combinations of spawner-recruit parameter options (e.g., changing R0 directly instead of the regime shift parameter). It is recommended to use block, trend or environmental effects on R0 only for long-term effects, and use time-vary effects on the regime shift parameter for transitory but multi-year deviations from R0.	
	
	\item If the R0 or steepness parameters are time-varying, then the model will use the current year's parameters to calculate the expected value of recruits as a function of the spawning biomass, then applies the recruitment deviations. If the regime shift parameter is time-varying, then the model applies the change in the regime shift parameter \textbf{after} calculating the expected value of recruits as a function of spawning biomass.

\end{itemize}

\hypertarget{ForecastTV}{}
\subsubsection[Forecast Considerations with Time-Varying Parameters]{\protect\hyperlink{ForecastTV}{Forecast Considerations with Time-Varying Parameters}}

Users should judiciously consider which parameter values are applied during forecast years. SS3 will default to use all base parameter values during the forecast period, but alternatively, which years of selectivity, relative F, and recruitment should be used during the forecast period by specifying in the \hyperlink{fore-specify}{forecast file}.

Time-varying parameters can extend into the forecast period. For example, a parameter with a time block that stops at the model end year will revert to the base parameter value for the forecast, but when the block definition extends to include some or all forecast years, the last block will apply to the forecast. A good practice is to use 9999 as the terminal year for the last block to ensure including all forecast years. If a parameter has deviations and the deviations' year range includes the forecast years, then the parameter will have process uncertainty in the forecast years and MCMC draws(if using) will include the variability.


\hypertarget{2DAR}{}
\subsection[Parameterizing the Two-Dimensional Autoregressive Selectivity]{\protect\hyperlink{2DAR}{Parameterizing the Two-Dimensional Autoregressive Selectivity}}
When the two-dimensional autoregressive selectivity feature is turned on for a fleet, the selectivity is calculated as a product of the assumed selectivity pattern and a non-parametric deviation term deviating from this assumed pattern:

\begin{equation}
\hat{S}_{a,t} = S_aexp^{\epsilon_{a,t}}
\end{equation}

where $S_a$ is specified in the corresponding age/length selectivity types section, and it can be either parametric (recommended) or non-parametric (including any of the existing selectivity options in SS3); $\epsilon_{a,t}$ is simulated as a two-dimensional first-order autoregressive (2D AR1) process:

\begin{equation}
vec(\epsilon) \sim MVN(\mathbf{0},\sigma_s^2\mathbf{R_{total}})
\end{equation}

where $\epsilon$ is the two-dimensional deviation matrix and $\sigma_s^2\mathbf{R_{total}}$ is the covariance matrix for the 2D AR1 process. More specifically, $\sigma_s^2$ quantifies the variance in selectivity deviations and $\mathbf{R_{total}}$ is equal to the kronecker product ($\otimes$) of the two correlation matrices for the among-age and among-year AR1 processes:

\begin{equation}
\mathbf{R_{total}}=\mathbf{R}\otimes\mathbf{\tilde{R}}
\end{equation}

\begin{equation}
\mathbf{R}_{a,\tilde{a}}=\rho_a^{|a-\tilde{a}|}
\end{equation}

\begin{equation}
\mathbf{\tilde{R}}_{t,\tilde{t}}=\rho_t^{|t-\tilde{t}|}
\end{equation}

where $\rho_a$ and $\rho_t$ are the among age and among year AR1 coefficients, respectively. When both of them are zero, $\mathbf{R}$ and $\mathbf{\tilde{R}}$ are two identity matrices and their Kronecker product, $\mathbf{R_{total}}$, is also an identity matrix. In this case selectivity deviations are essentially identical and mutually independent:

\begin{equation}
\epsilon_{a,t}\sim N(0,\sigma_s^2)
\end{equation} 

\myparagraph{Using the Two-Dimensional Autoregressive Selectivity}
Note, \citet{xu-new-2019} has additional information on tuning the 2D AR selectivity parameters. First, fix the two AR1 coefficients ($\rho_a$ and $\rho_t$) at 0 and tune $\sigma_s$ iteratively to match the relationship:

\begin{equation}
\sigma_s^2=SD(\epsilon)^2+\frac{1}{(a_{max}-a_{min}+1)(t_{max}-t_{min}+1)}\sum_{a=a_{min}}^{a_{max}}\sum_{t=t_{min}}^{t_{max}}SE(\epsilon_{a,t})^2
\end{equation}

The minimal and maximal ages/lengths and years for the 2D AR1 process can be freely specified by users in the control file. However, we recommend specifying the minimal and maximal ages and years to cover the relatively ``data-rich'' age/length and year ranges only. Particularly we introduce: 

\begin{equation}
b=1-\frac{\frac{1}{(a_{max}-a_{min}+1)(t_{max}-t{min}+1)}\sum_{a=a_{min}}^{a_{max}}\sum_{t=t_{min}}^{t_{max}}SE(\epsilon_{a,t})^2}{\sigma_s^2}
\end{equation}

as a measure of how rich the composition data is regarding estimating selectivity deviations. We also recommend using the Dirichlet-Multinomial method to ``weight'' the corresponding composition data while $\sigma_s$ is interactively tuned in this step.

Second, fix $\sigma_s$ at the value iteratively tuned in the previous step and estimate $\epsilon_{a,t}$. Plot both Pearson residuals and $\epsilon_{a,t}$ out on the age-year surface to check their 2D dimensions. If their distributions seems to be not random but rather be autocorrelated (deviation estimates have the same sign several ages and/or years in a row), users should consider estimating and then including the autocorrelations in $\epsilon_{a,t}$.

Third, extract the estimated selectivity deviation samples from the previous step for estimating $\rho_a$ and $\rho_t$ externally by fitting the samples to a stand-alone model written in Template-Model Builder (TMB). In this model, both $\rho_a$ and $\rho_t$ are bounded between 0 and 1 via applying a logic transformation. If at least one of the two AR1 coefficients are notably different from 0, the model should be run one more time by fixing the two AR1 coefficients at their values externally estimated from deviation samples. The Pearson residuals and $\epsilon_{a,t}$ from this run are expected to distribute more randomly as the autocorrelations in selectivity deviations can be at least partially included in the 2D AR1 process.

\hypertarget{continuous-seasonal-recruitment-sec}{}
\subsection[Continuous seasonal recruitment]{\protect\hyperlink{continuous-seasonal-recruitment-sec}{Continuous seasonal recruitment}}
Setting up a seasonal model such that recruitment can occur with similar and independent probability in any season of any year is awkward in SS3. Instead, SS3 can be set up so that each quarter appears as a year (i.e., a seasons-as-years model). All the data and parameters are set up to treat quarters as if they were years. Note that setting up a seasons-as-years model also requires that all rate parameters be re-scaled to correctly account for the quarters being treated as years.

Other adjustments to make when using seasons as years include:

\begin{itemize}
	\item Re-index all ``year seas'' inputs to be in terms of quarter-year because all are now season 1; increase end year (endyr) value in sync with this.
	\item Increase max age because age is now in quarters.
	\item In the age error definitions, increase the number of entries to reflect that age is now in quarters.
	\item In the age error definitions, re-code so that each quarter-age gets assigned to the correct age bin. This is because the age data are still in terms of age bins; i.e., the first 4 entries for quarter-ages 1 through 4 will all be assigned to age bin 1.5; the next four to age bin 2.5; you cannot accomplish the same result by editing the age bin values because the standard deviation of ageing error is in terms of age bin.
	\item In the control file, multiply the natural mortality age breakpoints and growth Amin and Amax values by 1/season duration.
	\item Decrease the R0 parameter starting value because it is now the average number of recruitments per quarter year.
	\item Edit the recruitment deviation (rec\_dev) start and end years to be in terms of quarter year.
	\item Edit any age selectivity parameters that refer to age, because they are now in terms of quarter age.
	\item If there needs to be some degree of seasonality to a parameter, then you could create a cyclic pattern in the environmental input and make the parameter a function of this cyclic pattern.
\end{itemize}

\pagebreak

\hypertarget{SS3Processes}{}
\section[Detailed Information on Stock Synthesis Processes]{\protect\hyperlink{SS3Processes}{Detailed Information on Stock Synthesis Processes}}

The processes and calculations within SS3 can be complex and not transparent based on the model input files. Here, additional information on processes within SS3 is provided.

\hypertarget{Jitter}{}
\subsection[Jitter]{\protect\hyperlink{Jitter}{Jitter}}
The following steps are now performed to determine the jittered starting parameter values (illustrated in Figure \ref{fig:jitter}):
\begin{enumerate}
	\item A normal distribution is calculated such that the pr(P\textsubscript{MIN}) = 0.1\% and the pr(P\textsubscript{MAX}) = 99.9\%.
	\item A jitter shift value, termed ``\textit{K}'', is calculated from the distribution equal to pr(P\textsubscript{CURRENT}).
	\item A random value is drawn, ``\textit{J}'', from the range of \textit{K}-jitter to \textit{K}+jitter.
	\item Any value which falls outside the 0-1 range (in the cumulative normal space) is mapped back from the bound to a point one-tenth of the way from the bound to the initial value.
	\item \textit{J} is a new cumulative normal probability value.
	\item Calculate a new parameter value, P\textsubscript{JITTERED}, such that pr(P\textsubscript{JITTERED}) = \textit{J}.
\end{enumerate}

\begin{figure}[ht]
	\begin{center}
		\includegraphics[alt={Illustration of the jitter algorithm.},scale = 0.75]{jitter_illustration}\\
		\caption{Illustration of the jitter algorithm.}
		\label{fig:jitter}
	\end{center}
\end{figure}

In SS3, the jitter fraction defines a uniform distribution in cumulative normal space +/- the jitter fraction from the initial value (in cumulative normal space). The normal distribution for each parameter, for this purpose, is defined such that the minimum bound is at 0.001, and the maximum at 0.999 of the cumulative distribution. If the jitter faction and original initial value are such that a portion of the uniform distribution goes beyond 0.001 or 0.999 of the cumulative normal, the new value is set to one-tenth of the way from the bound to the original initial value. 

Therefore, sigma = (max-min) / 6.18. For parameters that are on the log-scale, sigma may be the correct measure of variation for jitters, for real-space parameters, CV (= sigma/original initial value) may be a better measure. 

If the original initial value is at or near the middle of the min-max range, then for each 0.1 of jitter, the range of jitters extends about 0.25 sigmas to either side of the original value (though as the total jitter increases the relationship varies more than this), and the average absolute jitter is about half of that.  For values far from the middle of the min-max range, the resulting jitter is skewed in parameter space, and may hit the bound, invoking the resetting mentioned above. 

To evaluate the jittering, the bounds, and the original initial values, a jitter\_info table is available from \texttt{r4ss}, including sigma, CV and InitLocation columns (the latter referring to location within the cumulative normal - too close to 0 or 1 indicates a potential issue).

Note: parameters with min $\leq$ -99 or max $\geq$ 999 are not jittered to avoid unreasonable values (a warning is produced to indicate this).

\hypertarget{PriorDescrip}{}
\subsection[Parameter Priors]{\protect\hyperlink{PriorDescrip}{Parameter Priors}}
Priors on parameters fulfill two roles in SS3. First, for parameters provided with an informative prior, SS3 is receiving additional information about the true value of the parameter. This information works with the information in the data through the overall log likelihood function to arrive at the final parameter estimate. Second, diffuse priors provide only weak information about the value of a prior and serve to manage model performance during execution. For example, some selectivity parameters may become unimportant depending upon the values of other parameters of that selectivity function. In the double normal selectivity function, the parameters controlling the width of the peak and the slope of the descending side become redundant if the parameter controlling the final selectivity moves to a value indicating asymptotic selectivity. The width and slope parameters would no longer have any effect on the log likelihood, so they would have no gradient in the log likelihood and would drift aimlessly. A diffuse prior would then steer them towards a central value and avoid them crashing into the bounds. Another benefit of diffuse priors is the control of parameters that are given unnaturally wide bounds. When a parameter is given too broad of a bound, then early in a model run it could drift into this tail and potentially get into a situation where the gradient with respect that parameter approaches zero even though it is not at its global best value. Here the diffuse prior helps move the parameter back towards the middle of its range where it presumably will be more influential and estimable.  

The options for parameter priors are described as a function of $Pval$, the value of the parameter for which a prior is being calculated, as well as the parameter bounds in the case of the beta distribution ($Pmax$ and $Pmin$), and the input values for $Prior$ and $Pr\_SD$, which in some cases are the mean and standard deviation, but interpretation depends on the prior type. The Prior Likelihoods below represent the negative log likelihood in all cases.

\myparagraph{Prior Types}
Note that the numbering in v.3.30 is different from that used in v.3.24 (where confusingly -1 indicated no prior and 0 indicated a normal prior). The calculation of the negative log likelihood is provided below for each prior types, as a function of the following inputs:

\begin{tabular}{ll}
	$P_\text{init}$ & The value of the parameter for which a prior is being calculated where init can either be \\
	                & the initial un-estimated value or the estimated value (3rd column in control or \\
	                & control.ss\_new file)       \\
	$P_\text{LB}$   & The lower bound of the parameter (1st column in control file)     \\
	$P_\text{UB}$   & The upper bound of the parameter (2nd column in control file)     \\
	$P_\text{PR}$   & The input value for the prior input (4th column in control file)  \\
	$P_\text{PRSD}$ & The standard deviation input value for the prior (5th column in control file) \\
\end{tabular}

\begin{itemize}
	\item  \textbf{Prior Type = 0 = No prior applied} \\ 
	In a Bayesian context this is equivalent to a uniform prior between the parameter bounds.
	
	\item  \textbf{Prior Type = 1 = Symmetric beta prior} \\ 
	The symmetric beta is scaled between parameter bounds, imposing a larger penalty near the bounds. Prior standard deviation of 0.05 is very diffuse and a value of 5.0 provides a smooth U-shaped prior. The prior input is ignored for this prior type.
	\begin{equation} 
		\mu = -P_\text{PRSD} \cdot ln\left(\frac{P_\text{UB}+P_\text{LB}}{2} - P_\text{LB} \right) - P_\text{PRSD} \cdot ln(0.5)
	\end{equation}
	
	\begin{equation}
		\begin{split}
\text{Prior Likelihood} = &-\mu -P_\text{PRSD} \cdot ln\left(P_\text{init}-P_\text{LB}+0.0001\right) - \\
& P_\text{PRSD} \cdot ln\left(1-\frac{P_\text{init}-P_\text{LB}-0.0001}{P_\text{UB}-P_\text{LB}}\right)
		\end{split}
	\end{equation}

	\begin{figure}[ht]
		\begin{center}
			\includegraphics[alt={The shape of the symmetric beta prior across alternative standard deviation values and the change in the negative log-likelihood.},scale = 0.6]{SymetricBeta}\\
		\end{center}
		\caption{The shape of the symmetric beta prior across alternative standard deviation values and the change in the negative log-likelihood.}
	\end{figure}	

	
	\item \textbf{Prior Type = 2 = Beta prior} \\ 
	The definition of $\mu$ is consistent with CASAL's formulation with the $\beta_\text{PR}$ and $\alpha_\text{PR}$ corresponding to the $m$ and $n$ parameters.
	\begin{equation}
		\mu = \frac{P_\text{PR}-P_\text{LB}}{P_\text{UB}-P_\text{LB}} 
	\end{equation}
	\begin{equation}
		\tau  = \frac{(P_\text{PR}-P_\text{LB})(P_\text{UB}-P_\text{PR})}{P_\text{PRSD}^2}-1
	\end{equation}
	\begin{equation}
		\beta_\text{PR}  = \tau \cdot \mu
	\end{equation}
	\begin{equation}
		\alpha_\text{PR} = \tau (1-\mu)
	\end{equation}
	
	\begin{equation}
		\begin{split}
\text{Prior Likelihood} = &(1 - \beta_\text{PR}) \cdot ln(0.0001 + P_\text{init} - P_\text{LB}) + \\
&(1 - \alpha_\text{PR}) \cdot ln(0.0001 + P_\text{UB} - P_\text{init}) - \\
&(1 - \beta_\text{PR}) \cdot ln(0.0001 + P_\text{PR} - P_\text{LB}) - \\
&(1 - \alpha_\text{PR}) \cdot ln(0.0001 + P_\text{UB} - P_\text{PR})
		\end{split}
	\end{equation}	

	
	\item \textbf{Prior Type 3 = Log-normal prior} \\ 
	Note that this is undefined for $p <= 0$ so the lower bound on the parameter must be > 0. The prior value is input into the parameter line in natural log space while the initial parameter value is defined in normal space (e.g., init = 0.20, prior = -1.609438).
	\begin{equation}
		\text{Prior Likelihood} = \frac{1}{2} \left(\frac{ln(P_\text{init})-P_\text{PR}}{P_\text{PRSD}}\right)^2
	\end{equation}
	
	\item \textbf{Prior Type 4 = Log-normal prior with bias correction} \\ 
	This option allows the prior mean value to be entered as the ln(mean). Note that this is undefined for $p <= 0$ so the lower bound on the parameter must be > 0.
	\begin{equation}
		\text{Prior Likelihood} = \frac{1}{2} \left(\frac{ln(P_\text{init})-P_\text{PR} + \frac{1}{2}{P_\text{PRSD}}^2}{P_\text{PRSD}}\right)^2
	\end{equation}
	
	\item \textbf{Prior Type 5 = Gamma prior} \\ 
	The lower bound should be 0 or greater.
	\begin{equation}
		\text{scale} = \frac{{P_\text{PRSD}}^2}{P_\text{PR}}
	\end{equation}
	\begin{equation}
		\text{shape} = \frac{P_\text{PR}}{\text{scale}}
	\end{equation}
	\begin{equation}
		\text{Prior Likelihood} = -\text{shape} \cdot ln(\text{scale}) - ln\big(\Gamma(\text{shape})\big) + (\text{shape} - 1) \cdot ln(P_\text{init}) - \frac{P_\text{init}}{\text{scale}}
	\end{equation}
	
	\item \textbf{Prior Type 6 = Normal prior} \\ 
	Note that this function is independent of the parameter bounds.
	\begin{equation}
		\text{Prior Likelihood} = \frac{1}{2} \left(\frac{P_\text{init} - P_\text{PR}}{P_\text{PRSD}}\right)^2
	\end{equation}
\end{itemize}

%=========Forecast Module
\hypertarget{appendB}{}
\subsection[Forecast Module: Benchmark and Forecasting Calculations]{\protect\hyperlink{appendB}{Forecast Module: Benchmark and Forecasting Calculations}}\label{sec:forecast}

Stock Synthesis v.3.20 introduced substantial upgrades to the benchmark and forecast module. The general intent was to make the forecast outputs more consistent with the requirement to set catch limits that have a known probability of exceeding the overfishing limit. In addition, this upgrade addressed several inadequacies with the previous module, including:

\begin{itemize}
	\item The average selectivity and relative $F$ was the same for the benchmark and the forecast calculations;
	\item The biology-at-age in endyr+1 was used as the biology for the benchmark, but biology-at-age propagated forward in the forecast if there was time-varying growth;
	\item The forecast module had an inefficient approach to calculation of \gls{ofl} conditioned on previously catching \gls{abc};
	\item The forecast module implementation of catch caps was incomplete and applied some caps on a seasonally, rather than the more logical annual basis;
	\item The $F\text{mult}$ scalar for fishing intensity presented a confusing concept for many users;
	\item No provision for specification of catch allocation among fleets;
	\item The forecast allowed for a blend of fixed input catches and catches calculated from target $F$; this is not optimal for calculation of the variance of $F$ conditioned on a catch policy that sets \glsp{acl}.
\end{itemize}

The v.3.20 module addressed these issues by:
\begin{itemize}
	\item Providing for unique specification of a range of years from which to calculate average selectivity for benchmark, average selectivity for forecast, relative $F$ for benchmark, and relative $F$ for forecast;
	\item Create a new specification for the range of years over which to average size-at-age and fecundity-at-age for the benchmark calculation. In a setup with time-varying growth, it may make sense to do this over the entire range of years in the time series. Note that some additional quantities still use their endyr values, notably the migration rates and the allocation of recruitments among areas. This will be addressed shortly;
	\item Create a multiple pass approach that rectifies the \gls{ofl} calculation;
	\item Improve the specification of catch caps and implement specification of catch allocations so that there can be an annual cap for each fleet, an annual cap for each area, and an annual allocation among groups of fleets (e.g., all recreational fleets vs. all commercial fleets);
	\item Introduce capability to have implementation error in the forecast catch (single value applied to all fleets in all seasons of the year).
\end{itemize}

\myparagraph{Multiple Pass Forecast}
The most complicated aspect of the changes is with regard to the multiple pass aspect of the forecast. This multiple pass approach is needed to calculate both \gls{ofl} and \gls{abc} in a single model run. More importantly, the multiple passes are needed in order to mimic the actual sequence of assessment-management action - catch over a multi-year period. The first pass calculates \gls{ofl} based on catching \gls{ofl} each year, so presents the absolute maximum upper limit to catches. The second pass forecasts a catch based on a harvest policy, then applies catch caps and allocations, then updates the $F$'s to match these catches. In the third pass, stochastic recruitment and catch implementation error are implemented and SS3 calculates the $F$ that would be needed in order to catch the adjusted catch amount previously calculated in the second pass. With this approach, SS3 is able to produce improved estimates of the probability that $F$ would exceed the overfishing $F$. In effect it is the complement of the P* approach. Rather than the P* approach that calculates the stream of annual catches that would have an annual probability of $F > F\text{limit}$, SS3 calculates the expected time series of P* that would result from a specified harvest policy implemented as a buffer between $F\text{target}$ and $F\text{limit}$.

The sequence of multiple forecast passes is as follows:
\begin{enumerate}
	\item Pass 1 (a.k.a. Fcast\_Loop1)
	\begin{enumerate}
		\item Loop Years
		\begin{enumerate}
			\item SubLoop (a.k.a. ABC\_Loop) = 1
			\begin{enumerate}
				\item R = f(SSB) with no deviations
				\item $F$ = $F\text{limit}$
				\item Fixed input catch amounts ignored
				\item No catch adjustments (caps and allocations)
				\item No implementation error
				\item Result: \gls{ofl} conditioned on catching \gls{ofl} each year
			\end{enumerate}
		\end{enumerate}
	\end{enumerate}
	\item Pass 2
	\begin{enumerate}
		\item Loop Years
		\begin{enumerate}
			\item SubLoop = 1
			\begin{enumerate}
				\item R = f(SSB) with no deviations
				\item $F$ = $F\text{limit}$
				\item Fixed input catch amounts ignored
				\item No catch adjustments (caps and allocations)
				\item No implementation error
				\item Result: \gls{ofl} conditioned on catching \gls{abc} previous year. Stored in std\_vector.
			\end{enumerate}
			\item SubLoop = 2
			\begin{enumerate}
				\item R = f(SSB) with no deviations
				\item $F$ = $F\text{target}$ to get catch for each fleet in each season
				\item Fixed input catch amounts replace catch from step 2
				\item Catch adjustments (caps and allocations) applied on annual basis (after looping through seasons and areas within this year). These adjustments utilize the logistic joiner approach so the overall results remain completely differentiable.
				\item No implementation error
				\item Result: \gls{abc} as adjusted for caps and allocations
			\end{enumerate}
			\item SubLoop = 3
			\begin{enumerate}
				\item R = f(SSB) with no deviations
				\item Catches from Pass 2 multiplied by the random term for implementation error
				\item $F$ = adjusted to match the catch*error while taking into account the random recruitments. This is most easily visualized in a \gls{mcmc} context where the recruitment deviation and the implementation error deviations take on non-zero values in each instance. In \gls{mle}, because the forecast recruitments and implementation error are estimated parameters with variance, their variance still propagates to the derived quantities in the forecast.
				\item Result: Values for $F$, \gls{ssb}, Recruitment, Catch are stored in std-vectors
				\begin{itemize}
					\item In addition, the ratios $F$/$F\text{limit}$ and \gls{ssb}/\gls{ssb}limit or \gls{ssb}/\gls{ssb}target are also stored in std\_vectors.
					\item Estimated variance in these ratios allows calculation of annual probability that $F > F\text{limit}$ or B < Blimit. This is essentially the realized P* conditioned on the specified harvest policy.
				\end{itemize}
			\end{enumerate}
		\end{enumerate}
	\end{enumerate}
\end{enumerate}

\myparagraph{Example Effects on Correlations}
An example that illustrates the above process was conducted. The situation was a low M, late-maturing species, so changes are not dramatic. The example conducted a 10-year forecast and examined correlations with derived quantities in the last year of the forecast. This was done once with the full set of 3 passes as described above, and again with only 2 passes and stochastic recruitment occurring in pass 2, rather than 3. This alternative setup is more similar to forecasts done using previous model versions.

\begin{center}
	\begin{longtable}{p{0.4cm} p{2.75cm} p{3cm} p{1cm} p{0.4cm} p{2.75cm} p{2cm} p{1cm}}
		\hline
		 & \multicolumn{3}{l}{2 Forecast Passes with $F$ from} & & \multicolumn{3}{l}{2 Forecast Passes with catch from} \\
		 & \multicolumn{3}{l}{\gls{abc} and random recruitment} & & \multicolumn{3}{l}{target $F$ and equilibrium recruitment} \\
		\hline
		 & Factor X & Factor Y & Corr & & Factor X & Factor Y & Corr \\
		\hline
		A1 & F 2011 & RecrDev 2002 & -0.126 & A2 & F 2011 & RecrDev 2002 & 0.090 \\
		B1 & F 2011 & Recr 2002 & 0.312 & B2 & F 2011 & Recr 2002 & 0.518 \\
		C1 & ForeCatch 2011 & RecrDev 2002 & 0.000 & C2 & ForeCatch 2011 & RecrDev 2002 & 0.129 \\
		D1 & ForeCatch 2011 & Recr 2002 & 0.455 & D2 & ForeCatch 2011 & Recr 2002 & 0.555 \\
		\hline		
	\end{longtable}
\end{center}

Correlation A2 shows a small positive correlation between the recruitment deviation in 2002 and the $F$ in 2011. This is probably due to the fact that a positive deviation in recruitment in 2002 will reduce the chances that the biomass in 2011 will be below the inflection point in the control rule. This occurs because in calculating catch from $F$, the model effectively ``knows'' the future recruitments. I predict that this B1 correlation would be near zero if there was no inflection in the control rule.

Correlation A1 shows this turning into a negative correlation. This is because the future catches are first calculated from equilibrium recruitment, then when random recruitments are implemented, a positive recruitment deviation will cause a negative deviation in the $F$ needed to catch that now ``fixed'' amount of future catch.

Correlations B1 and B2 are in terms of absolute recruitment, not recruitment deviation. Now overall model conditions that cause a higher absolute recruitment level will also result in a higher forecast level. No surprise there, and the correlation is stronger when variance is based on catch is calculated from $F$ (B2).

Correlation C2 shows a positive correlation between recruitment deviation in 2002 and forecast catch in 2011. However, correlation C1 is 0.0 because the forecast catch in 2011 is set based on equilibrium recruitment and is not influenced by the recruitment deviations.

\myparagraph{Future Work}
\begin{itemize}
	\item More testing with high M, rapid turnover conditions
	\item Testing without inflection in control rule
	\item Consider separating implementation error into a pass \#4 so results will more clearly show effect of assessment uncertainty separate from implementation uncertainty
	\item Consider adding a random ``assessment'' error which essentially is a random variable that scales population abundance before passing into the forecast stage. Complication is figuring out how to link it to the correlated error in the benchmark quantities
	\item Because all of these calculations occur only in the standard deviation phase (sdphase) or the \gls{mcmc} evaluation (mceval) phase, it would be feasible for mceval calls to add a pass that is implemented many times and in which random forecast recruitment draws are made.
	\item Factors like selectivity and fleet relative F levels are calculated as an average of these values during the time series. This is internally consistent if these factors do not vary during the time series (although clearly this is a stiff model that will underestimate process variance). However, if these factors do vary over time, then the average used for the forecast will under-represent the variance. A better approach would be to set up the parameters of selectivity as a random process that extends throughout the forecast period, and to update estimated selectivity in each year of the forecast based upon the random realization of these parameters.
\end{itemize}

	


%=========F mortality in SS3
% \subsection{Fishing Mortality in Stock Synthesis}

The implementation and reporting of fishing mortality rate, $F$, in SS3 has some aspects that are more complex than in simpler models. This description provides an overview of the ways in which $F$ is calculated, used, and reported.  

\myparagraph{Rationale}
Fishery management systems expect to have a measure of annual fishing mortality ($F$) that describes the intensity of the fishery such that an optimal level of $F$ can be articulated and accountability measures can be invoked if $F$ is too high, e.g., overfishing. This concept is simple and straightforward if the model is a simple biomass dynamics model such that a single annual $F$ value operates on the entirety of a non-age structured population. It also is simple for age-structured models that have a single fishing fleet and knife-edge selectivity beginning at some specified age. The simplicity of $F$ disappears quickly as models invoke a variety of realistic complexities such as: allowing the $F$ to differ among ages or to be based on size; using a collection of fleets with different $F$ levels and different age patterns for $F$; spreading the population across areas and allowing different fleets with different $F$ among the areas. An unambiguous measure of annual fishing intensity that represents the cumulative effect of all that complexity is a challenge. This problem has not been solved with SS3, but some logical alternatives have been made available.

\myparagraph{Nomenclature}

The quantities associated with $F$ calculations are defined as:

$f$ is fleet.

$t$ is a time step; continuous across years $y$ and seasons $s$; equivalent to year if only 1 season.

$a$ is age.

$C_{t,f}$ is fleet-specific catch in a time step.

$s_{t,f,a}$ is age-specific selectivity for a fleet. If selectivity is length-specific, then age-specific selectivity due to length-selectivity is calculated as the dot product across length bins of length selectivity and the normal (or lognormal) distribution of length-at-age. If selectivity is both length- and age-based, which is an entirely normal concept in SS3, then age selectivity due to length selectivity is calculated first, then multiplied by the direct age selectivity. This compound age selectivity is used in the mortality calculations and is reported as $Asel2$ in report:32 of report.sso. See appendix to \citet{methotstock2013} for more detail on this.

Selectivity can be sex-specific, and different growth morphs and platoons can have different age-selectivity due to the effect of length-selectivity on their unique size-at-age. This added dimension, $g$, for biological group is not included in the nomenclature here but exists in all the SS3 calculations.

$B_{t,f}$ is fleet specific available biomass, e.g., total biomass filtered by fleet-specific age selectivity, $s_{t,f,a}$. Note that this is not adjusted by the $max(s_{t,f,a})$.

$F_{t,f}'$ is a fleet's fishing mortality for the age that has selectivity equal to 1.0. This is termed F\_scalar or  $F'$ in the SS3 system. If your model is using $F'$s as parameters, then the parameter values are for the $F'$. Note that some selectivity curves, like double normal, are explicit about having a maximum of 1.0. But other curves like logistic and combinations of length-selectivity and growth, may produce an age-selectivity curve that never reaches 1.0 and some situations, especially 2DAR, will produce selectivity >1.0 routinely. In all cases, the resultant $F_{t,f,a}$ comes from $F_{t,f}' * s_{t,f,a}$. $F_{t,f,a}$ is output in report:32 with the rows labelled simply as "F". The reported $F'$ values are never rescaled to be the $F$ for the age with peak selectivity. Users need to take this into account if they are comparing reported $F'$ values to reported vector of true $F_{t,f,a}$ values.

Apical selectivity is the maximum age-specific selectivity and is not explicit in any internal calculation in SS3, it is just for reporting. If selectivity has a maximum value of 1.0, then apical\_F and F\_scalar are identical. You can find apical selectivity reported as maximum\_ASEL2, immediately after report:32.

\myparagraph{$F$ Calculation}
SS3 allows for three approaches to estimate the $F'$ that will match the input values for retained catch. Note that SS3 is calculating the $F'$ to match the retained catch conditional on the fraction of total catch that is retained, e.g., the total catch can be partitioned into retained and discarded portions.

\begin{enumerate}
	\item Pope's method decays the numbers-at-age to the middle of the season, calculates a harvest rate for each fleet, $H_{t,f}$, that is the ratio of $C_{t,f}$ to $B_{t,f}$, then decays the survivors to the end of the season. The total mortality, $Z_{t,a}$, from the ratio of survivors to initial numbers, is then calculated. The $Z$ is subsequently used for in-season interpolation to get expected values for observations.
	
	\item $F$ as parameters method uses the standard Baranov catch equation and lets ADMB find the $F'$ parameter values that produce the lowest negative log-likelihood, which includes fit to the input catch data. $F$ as parameters method tends to work better than Pope's method or hybrid $F$ method in high $F$ situations because it allows for some lack of fit to catch levels in early iterations and can later improve this fit as it closes in on the best solution.
	
	\item Hybrid $F$ method starts by calculating a harvest rate, $H$, using Pope's method, then converts these $H$ values, which have units of fractional harvest rate, into an approximate of $F'$ in exponential units, tuning these $F'$ values over a few iterations to get a better match to each fleet's catch.
\end{enumerate}

Items to note:
\begin{itemize}
	\item SS3 includes a permutation on the $F$ as parameters method. In the first few phases, SS3 uses the hybrid $F$ method, then between phases it converts these directly calculated $F'$ values into parameters and proceeds in subsequent phases and MCMC to use the parameter approach. This variation on the parameter method is the recommend approach in high $F$ situations.
	
	\item With Pope's method, the $H$ values are fraction caught, so duration of the season does not matter. The $F$ as parameters and hybrid $F$ method treat $F'$ identically and multiply the $F'$ values by season duration (which has units of fraction of a year) as it is used. Each of the $F$ methods ends up with a $Z_{t,f}$ that is used for in-season interpolation.
\end{itemize}

\myparagraph{Annual\_F and F\_std reporting}
SS3 provides several options for reporting overall annual fishing intensity. $F\text{std}_y$ is an ADMB-specific derived quantities, so its variance is calculated. Annual\_F (annF) is a building block for F\_std and provides additional reporting of some of the same quantities.

The options for F\_std reporting are in the starter.ss file:
\begin{center}
	\begin{longtable}{p{2cm} p{12cm}}
		\hline
		5 & \# F\_std\_reporting\_units: \Tstrut\\
		  & 0 = skip; \\
		  & 1 = exploitation(Bio); \\
		  & 2 = exploitation(Num); \\ 
		  & 3 = sum($F$ scalars*seas); \\
		  & 4 = mean $F$ for range of ages (numbers weighted); \\
		  & 5 = unweighted mean $F$ for range of ages. \Bstrut\\
		\hline
		3 7 & \# min and max age over which mean $F$ will be calculated with F\_reporting = 4 or 5 \Tstrut\Bstrut\\
		\hline
	\end{longtable}
	\vspace*{-1.7\baselineskip}
\end{center}

\begin{itemize}
	\item For options 1 and 2, the numerator is retained catch numbers or biomass, and the denominator is summary numbers or summary biomass.
	\item For option 3, the result is simply the season weighted sum of the $F$'s (except no season weighting if using Pope's harvest rate approach). Option 3 will be misleading in a multi-area model as the calculation is insensitive to the proportion of the population in the area where the F is being applied.
	\item For options 4 and 5, the $F$ is calculated as $Z-M$ where $Z$ is calculated as $ln(N_{t+1,a+1}/N_{t,a})$, thus $Z$ subsumes the effect of $F$. In a multi-area model, the N values are summed across areas so counters the shortcoming of option 3, but the reported value is buffered if there in a large portion of the population in a lightly fished area.
	\item The F\_std is calculated for each year of the estimated time series and of the forecast. Additionally, a value with the same units is calculated in the benchmark calculations to provide a basis for scaling the output. These benchmark values are reported in the Mgmt\_Quantity section of derived quantities with labels like annF\_Btgt. 
	\item Prior to v.3.30.15, these quantities were inaccurately labeled Fstd\_Btgt.
\end{itemize}

F\_std scaling is selected in the starter file as:
\begin{center}
	\begin{longtable}{p{2cm} p{12cm}}
		%\multicolumn{2}{l}{The starter file line:}\\
		\hline
		0 & \# F\_std\_scaling: \Tstrut\\
		& 0 = no scaling; \\
		& 1 = $F$ / $F_{SPR}$; \\ 
		& 2 = $F$ / $F_{MSY}$; \\
		& 3 = $F$ / $F_{BTGT}$.\Bstrut\\
		\hline
	\end{longtable}
	\vspace*{-1.7\baselineskip}
\end{center}

Note that $F$ means annual F\_std, $F_{MSY}$ means F\_std at MSY.

The results of these calculations is displayed most explicitly in the report.sso table "EXPLOITATION report:14". Here, the table columns are:

Yr Seas Seas\_dur F\_std annual\_F annual\_M <each fleet's F\_scalar>

In this table, the displayed value for annual\_F will be from the $F=Z-M$ method regardless of which option was chosen for F\_std.  If F\_std uses option 4 or 5, then the annual\_F will use the same range of ages. Otherwise, annual\_F will be for the age that is the mid-age of the age range.

% While F is a simple concept in a biomass dynamics model with a single fishing fleet, the concept of "F" as a single number is very incomplete when there are multiple fleets, some with length-based and/or dome-shaped selectivity. In SS3, the F multiplier is multiplied first by fleet-specific relative F's to get the F for each fleet, then that fleet-specific F is multiplied by the age-selectivity to get F at age. Depending on the choice of selectivity pattern, these age-specific F's may or may not peak at 1.0 and can even exceed 1.0 in some circumstances. The realized F-at-age for each fleet is (F multiplier) * (relative F) * (age-selectivity). You can see the results in the F\_AT\_AGE section of the Report file. 

% While F multiplier is always done as just described, the user has various options for the reporting of the realized F. These are in starter.ss and described in the manual.

\myparagraph{Relative $F$ and Fmult}
The $F'$ is fleet-specific, so it is useful to have a concept of relative $F$, $\text{rel}F_f$, among fleets. In SS3, $\text{rel}F_f= F_{t,f}'/\sum_{f}^{}F_{t,f}'$ for a single time period $t$. In the benchmark and forecast routines, SS3 can calculate $\text{rel}F_f$ using $F_{t,f}'$ over a range of years, or the user can input custom $\text{rel}F$ values for benchmark and forecast in the forecast.ss file. Note that in a multi-season model setup, $\text{rel}F_f$ is implemented as $\text{rel}F_{s,f}$ where $s$ is the season. These get multiplied by season duration as they are used.

In the benchmark section of the code, SS3 searches for an Fmult to achieve various management reference points (often referred to as benchmarks). In this search, SS3 calculates a benchmark $F$ as  $F_{ben,f}' = F\text{mult} * \text{rel}F_f$, then calculates equilibrium yield and spawning biomass per recruit (SPR). SS3 searches for the Fmult that satisfies the search conditions, first for user-specified SPR, then for user-specified spawning biomass at a management target (B\textsubscript{TGT} or $F_{0.1}$), then for MSY. The resultant benchmark quantities are reported in the derived quantities, but Fmult and $F_{ben,f}'$ are only reported in the Forecast\_report.sso file. SS3 stores the benchmark Fmult values so that user can invoke them for the forecast.

\myparagraph{Units for Stock Synthesis inputs related to $F$}
Below is a list of items to consider in terms of units for $F$ in SS3:
\begin{itemize}
	\item If F\_ballpark is specified in the control.ss file, its units are the same as annF, so is not fleet-specific.
	
	\item $F$ as parameter values has units of fleet-specific apical $F'$.
	
	\item In the forecast.ss file there is an option to input a vector of relF values. These are dimensionless and will be rescaled to sum to 1.0.
	
	\item In the forecast.ss file there is an option to specify an $F$ scalar for the forecast. The units of $F$ scalar are the same as the Fmult values calculated in benchmark. There are a full set of options for forecast $F$ scalar that can be selected in the forecast file 
	%(-1 = none; 0 = simple; 1 = F\textsubscript{SPR}; 2 = F\textsubscript{MSY} 3 = F\textsubscript{BTGT} or F\textsubscript{0.1}; 4 = Ave F (uses first-last relative F years); and 5 = input annual F scalar). 
	If the forecast $F$ scalar is set as $F_\text{SPR}$, then SS3 will use SPR\_Fmult calculated in benchmark and reported in Forecast-report.sso. If user selects the option to input an annual $F$ scalar, option 5, then the value is input on a following line. Whichever method the user selects for forecast $F$ scalar (Fmult), SS3 will start the forecast by creating a fleet-specific vector of apical $F$ values from Fmult*rel$F_f$.
	
	\item Also in the forecast.ss file, the last section of inputs allows for input of time and fleet specific apical $F_{t,f}'$ values that override the basic forecast $F$ specification described above.
\end{itemize}


\pagebreak