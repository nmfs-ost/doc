\hypertarget{DataWeight}{}
\subsection[Data Weighting]{\protect\hyperlink{DataWeight}{Data Weighting}}\label{sec:DataWeight}


In 2015 there was a CAPAM workshop dedicated to data weighting. Description of the workshop can be found on the \href{http://capamresearch.org/data-weighting/workshop}{CAPAM} website. The presentations from the workshop are available through that website and many of them were included in a special issue of \href{https://sciencedirect.com/journal/fisheries-research/vol/192}{Fisheries Research}.

Currently, there are three main methods for weighting length and data applied for U.S. West Coast assessments using Stock Synthesis.

\begin{enumerate}
	\item \hyperlink{MI}{\textbf{McAllister - Ianelli}}: Effective sample size is calculated from fit of observed to expected length or age compositions. Tuning algorithm is intended to make the arithmetic mean of the input sample size equal to the harmonic mean of the effective sample size \citep{mcallister-bayesian-1997}.
	
	\item \hyperlink{Francis}{\textbf{Francis}}: Based on variability in the observed mean length or age by year, where the sample sizes are adjusted such that the fit of the expected mean length or age should fit within the uncertainty intervals at a rate which is consistent with variability expected based on the adjusted sample sizes (Method ``TA1.8'') \citep{francis-data-2011}
	
	\item \hyperlink{DM}{\textbf{Dirichlet-Multinomial}}: A new likelihood (as opposed to the standard multinomial) which includes an estimable parameter (theta) which scales the input sample size. In this case, the term ``Effective sample size'' has a different interpretation than in the McAllister-Ianelli approach \citep{thorson-model-based-2017}.
\end{enumerate}

\myparagraph{Applying the methods}

\hypertarget{MI}{}
\myparagraph{McAllister-Ianelli}
The ``Length\_Comp\_Fit\_Summary'' and ``Age\_Comp\_Fit\_Summary'' sections in the Report file include information on the harmonic mean of the effective sample size and arithmetic mean of the input sample size used in this tuning method. In the r4ss package, these tables are returned by the \texttt{SS\_output()} function as \texttt{ \$Length\_comp\_Eff\_N\_tuning\_check} and \texttt{ \$Age\_comp\_Eff\_N\_tuning\_check}.

A convenient way to process these values into the format required by the control file is to use the function:

\texttt{SS\_tune\_comps(replist, option = ``MI'')}

where the input ``replist'' is the object created by \texttt{SS\_output()}. This function will return a table and also write a matching file called \texttt{suggested\_tuning.ss} to the directory where the model was run.

For version 3.30 models, the table created by \texttt{SS\_tune\_comps()} can be pasted into the bottom of the control file in the section labeled ``Input variance adjustments'', followed by the terminator line which indicates the end of the section. 


Also see the help page for the \texttt{r4ss} function \texttt{SS\_varadjust()} which can be used to automatically write a new control file if you want to streamline the process of applying multiple iterations of this tuning method.

If the tuning has been implemented, the green lines in the figure below would approximately intersect at a point which is on the black 1-to-1 diagonal line in this figure created by the \texttt{r4ss} function \texttt{SS\_plots()}.

\begin{figure}[ht]
	\begin{center}
		\includegraphics[alt={The relationship between the observed sample size (the input sample number) versus the effective sample size where the effective sample size is the product of the input sample size and the data weighting applied to the data set.},scale = 0.65]{appendixB_McAllister_Ianelli}\\
	\end{center}
	\caption{The relationship between the observed sample size (the input sample number) versus the effective sample size where the effective sample size is the product of the input sample size and the data weighting applied to the data set.}
	\label{(fig:mcallister)}
\end{figure}

There are a couple of challenges posed by the McAllister-Ianelli data-weighting approach:
\begin{enumerate}
	\item Subjective choice of how many iterations to take to achieve adequate convergence. Often just one iteration is applied.
	
	\item Takes time to implement so tuning is rarely repeated during retrospective or sensitivity analyses.
\end{enumerate}

\hypertarget{Francis}{}
\myparagraph{Francis}

Implementation: recommended adjustments are calculated by the \texttt{r4ss} functions \texttt{SSMethod.TA1.8()} and \texttt{SSMethod.Cond.TA1.8()}. These functions are rarely used alone but are called by the \texttt{SS\_plots()} function when making plots like the one below. For version 3.30 models, the simplest way to get the adjustments in the format required by the control file is to use the \texttt{SS\_tune\_comps()} function (described above under the McAllister-Ianelli method), but with a different option specified: 

\texttt{SS\_tune\_comps(replist, option = ``Francis'')}

The figure below shows the estimated 95\% intervals around the observed mean length by year based on the input sample size (thick lines) and the increase in that uncertainty which would occur if the sample sizes were adjusted according to the proposed multiplier.

\begin{figure}[ht]
	\begin{center}
		\includegraphics[alt={The mean length of the length samples for each year from the MexCal S1 NSP fleet with 95\% confidence intervals based on current samples sizes using the Francis data weighting method (referred to as TA1.8). Thinner intervals with capped ends show result of further adjusting sample sizes based on the suggested multiplier, 0.2739, with 95\% intervals for length data from the fleet. The blue line shows the expected variation of the mean length across years.},scale = 0.15]{appendixB_Francis}\\
	\end{center}
	\caption{The mean length of the length samples for each year from the MexCal S1 NSP fleet with 95\% confidence intervals based on current samples sizes using the Francis data weighting method (referred to as TA1.8). Thinner intervals with capped ends show result of further adjusting sample sizes based on the suggested multiplier, 0.2739, with 95\% intervals for length data from the fleet. The blue line shows the expected variation of the mean length across years.}
	\label{fig:francis}
	%\figurename{name}
\end{figure}

There are a several of challenges posed by the Francis data-weighting approach:
\begin{enumerate}
	\item Subjective choice of how many iterations to take to achieve adequate convergence. Often, just one iteration is applied.
	
	\item Takes time to implement so tuning is rarely repeated during retrospective or sensitivity analyses.
	
	\item Recommended adjustment can be sensitive to outliers (remove a few years of anomalous composition data can lead to large change in recommended adjustment).
\end{enumerate}

Finally, in simulation work comparing both the Francis and McAllister and Ianelli data weighting approaches, indicate that the arithmetic averaging of effective samples sizes from the McAllister and Ianelli approach was inferior to other methods \citep{punt_insights_2016}.

\hypertarget{DM}{}
\myparagraph{Dirichlet-multinomial}

The Dirichlet-Multinomial should only be used when there is a
substantive basis for setting an approximate value for input sample
size.  This is important because: 

\begin{itemize}

\item The input-sample-size provided by the user is an upper bound on
weighting for those data, such that a dummy value of 1 will cause those
data to never be assigned a weight greater than 1.

\item Changes in input sample size are not quite proportionally
offset by changes in the estimated weighting parameter, such that
using input-sample-size of 100 vs. 200 will result in small differences
in estimated data weights (this property is fixed by the
multivariate-Tweedie likelihood, where changes in input-sample-size are
exactly offset by changes in the estimated weighting parameters).   

\end{itemize}

To develop an input-sample-size, Jim Thorson recommends:

\begin{enumerate}

\item Using nonparametric (bootstrapping), design-based, or model-based
estimators to identify the variance of expanded compositional data, and
then deriving an input-sample-size from that; 

\item Using \textit{a priori} reasoning, e.g., about fishery compositions
deserving a lower input-sample-size than a survey in cases when the
input-sample-size for the survey is known, but that for the fishery is
unknown. 

\item If the proceeding options are not feasible, using accepted
standards in that region and review context, i.e., assigning
input-sample-size of arbitrary value X if that is what has always been
done in that region/stock, there's no basis for changing it, and it is
greater than or equal to the likely effective sample size; 

\end{enumerate}

Change the choice of likelihood and set parameter choices in the data
file: 

\begin{itemize}
\item In the specification of the length and/or age data, change
``CompError'' column in age and length comp specifications from 0 to 1 and
``ParmSelect'' from 0 to a sequence of numbers from 1 to N where N is the
total number of combinations of fleet and age/length. Note that there
cannot be any skipped values from 1 to N, otherwise the model will exit
on error while reading the input files. 
	
\item Resulting input should look similar to:
\begin{small}
\begin{verbatim}
#_mintail addtocomp combM+F CompressBins CompError ParmSelect minsamplesize   
-1        0.001     0       0            1         1          0.001 #_fleet:1
-1        0.001     0       0            1         2          0.001 #_fleet:2
\end{verbatim}
\end{small}

\item The ParmSelect column can also have repeated values so that
multiple fleets share the same log(Theta) parameter. 
	
\item If you have both length and age data, the ParmSelect should have
separate numbers for each, e.g., 1 and 2 for the length comps and 3 and 4
for the age comps for the same two fleets. 
	
\end{itemize}

Add parameter lines to the control file:

\begin{itemize}
	\item Add as many parameter lines as the maximum numbers in the ParmSelect column. The new parameter lines go after the main selectivity parameters but before any time-varying selectivity parameters
	
\item Given that the input-sample-size is fixed at a reasonable level,
Jim Thorson recommends bounds of -5 to 5, with a starting value of 0
(which corresponds to a weight of about 50\% of the input sample size).
However, parameter estimates above 5 are associated with 99-100\% weight
with little information in the likelihood about the parameter value.
Therefore, an upper bound of 5 may help identify cases that otherwise
would have convergence issues.  When the parameter hits this upper
bound, consider either switching those data to a multinomial (which
behavior the DM is asymptotically approaching) or using a Bayesian prior
/ likelihood penalty to ensure that the estimated parameter does not
rest on this boundary.  	


\item In consultation with Jim Thorson, Ian Taylor proposed a normal
$N(0, 1.813)$ prior for the \texttt{ln(DM\_parm)} parameters which to
counteract the effect of the logistic transformation between this
parameter and the data weighting. The 1.813 value was calculated as the
standard deviation of the distribution of $log(\theta)$ values derived
from starting with a uniform distribution on the weights, weight =
$\theta/(1+\theta) \sim U(0,1)$, and solving for $log(\theta)$. This
prior was applied to the 2020 Pacific Hake stock assessment
\citep{grandin-status-2020} and led to better MCMC convergence with
relatively little impact on the maximum likelihood estimates. 
	
\item Example parameter lines are below (columns 8-14 not shown): 
    \begin{small}
      % minipage to stop page break that was falling within these lines
      \begin{minipage}{\linewidth} 
\begin{verbatim}
  #_LO  HI  INIT  PRIOR PR_SD PR_type PHASE ... # Parm_name
  -5    20  0     0     1.813 6       2     ... # ln(DM_theta)_1
  -5    20  0     0     1.813 6       2     ... # ln(DM_theta)_2
\end{verbatim}
      \end{minipage}
    \end{small}
	
\item Reset any existing variance adjustments factors that might have been used for the McAllister-Ianelli or Francis tuning methods. In v.3.24 this means setting the values to 1, in v.3.30, you can simply delete or comment-out the rows with the adjustments.
\end{itemize}

The \texttt{SS\_output} function in r4ss returns table like the following:
\begin{small}
\begin{verbatim}
  $Dirichlet_Multinomial_pars
                     Value Phase Min Max     Theta Theta/(1+Theta)
  ln(DM_theta)_1 -0.164022     2  -5  20 0.8487233       0.4590862
  ln(DM_theta)_2  2.246280     2  -5  20 9.4525070       0.9043292
\end{verbatim}
\end{small}


The ratio shown in the final column is the estimated multiplier which can be compared to the sample size adjustment estimated in the other tuning methods above (the New\_Var\_adj column in the table produced by the \texttt{SS\_tune\_comps} function in r4ss).

If the reported $\theta/(1+\theta)$ ratio is close to 1.0, that indicates that the model is trying to tune the sample size as high as possible. In this case, the $ln(\theta)$ parameters should be fixed at a high value, like the upper bound of 20, which will result in 100\% weight being applied to the input sample sizes. An alternative would be to manually change the input sample sizes to a higher value so that the estimated weighting will be less than 100%.

Note that a constant of integration was added to the Dirichlet-multinomial likelihood equation in v.3.30.17. This will change the likelihood value, but parameter estimates and expected values should remain the same as in previous versions of SS3.

Some challenges posed by the Dirichlet-multinomial data-weighting approach:
\begin{enumerate}
	\item Does not allow weights above 100\% (by design) so it is not yet clear how best to deal with the situation when the estimated weight is close to 100\%.
	\item Parameterization has potential to cause convergence issues or inefficient MCMC sampling when weights are close to 100\% if no prior is applied as discussed above.
\end{enumerate}
